% Report-ConclusionsFutureWork.tex
% Simon Hulse
% simonhulse@protonmail.com
% Last Edited: Wed 27 Nov 2024 02:30:08 PM EST

% !TeX root = ./Report-Body.tex
\chapter{Conclusions and Future Work}

\section{Conclusions}
In this thesis, a novel relaxation model, describing I$_3$S spin-systems has
been presented. This model is applicable to all molecular motion regimes, which
is one of its stand-out advantages over other models presented. As well as
this, via incorporation of I-spin CSA contributions, it is well-equipped to
describe relaxation in $^{13}$CF$_3$ spin systems.

Utilising this theory, calculations comparing relaxation phenomena in $^{13}$CH$_3$ and $^{13}$CF$_3$ moieties predict that there are significant advantages in using trifluoromethyl moieties over methyls. This is on account of the slower relaxation rates of the coherences considered. Favourable behaviour in $^{13}$CF$_3$ can largely be accounted for by a geometry-driven reduction in the magnitude of dipolar couplings between spins. The $^{19}$F CSA is also capable of having a noticeable impact on rates as well.

An attempt to mimic a protein using our test system had limited success. Via hydrodynamic considerations, it was ascertained that the test molecule was able to move very rapidly though the glycerol medium. The motion it adopts is highly anisotropic, with its translational behaviour being 'bullet-like'. Nonetheless, fits of our proposed relaxation models to the rate data obtained illustrate that we have a strong handle on relaxation. Our investigation supports a Woessner-type hopping combined with anisotropic rotation of our test molecule, due to the seeming lack of temperature dependence on methyl tumbling.

\section{Future Work}
The results of the calculations presented in this thesis provides motivation to proceed with the development of an NMR methodology that uses $^{13}$CF$_3$ moieties as probes in large biomolecules. For such a methodology to be realised, further work will still be required in two principle areas.

In order for a $^{13}$CF$_3$ TROSY experiment to become a well-established technique, a method of incorporating this spin-label into proteins is necessary. Work on this is currently being undertaken by research groups in collaboration with the Badlwin group, with promising results starting to emerge\cite{RN55,RN19}.

Establishing a rigorous understanding of relaxation in I$_3$S spins systems has opened up the possibility to evaluate a pulse sequence's effectiveness on $^{13}$CF$_3$ moieties. Via density matrix calculations, simulations of NMR experiments will provide a route to determining an optimal $^{13}$CF$_3$ TROSY experiment. This will hopefully enable studies of biomolcules that are currently out of our grasp using NMR.
