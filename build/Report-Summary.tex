% !TeX root = ./Report-Body.tex
\chapter{Summary}%

% Write the Summary text inside the summarytext environment.

% Put the Summary references in PartII-SummaryReferences.bib

\begin{summarytext}
NMR spectroscopy is a highly valuable technique for the characterisation of structure and dynamics of biomolecules. Despite this, the effects of relaxation limits the size of molecules that can be studied. Transverse Relaxation Optimised Spectroscopy (TROSY) is a method which has increased this limit, though there is a desire to push it even further.\\
We believe that a TROSY experiment which probes $^{13}$CF$_3$ moieties could allow studies of macromolecules which are currently deemed too large, in the context of solution-state NMR. Such systems include membrane-bound proteins and virus particles.\\
A novel relaxation theory is presented that is well-suited to describe $^{13}$CF$_3$ systems. Comparisons of relaxation behaviour between methyl and $^{13}$CF$_3$ groups were made using our theory. Our results indicate that magnetisations associated with $^{13}$CF$_3$ moieties relax at noticeably slower rates. This observation was determined to be driven largely by dipolar coupling effects, with the $^{19}$F chemical shift anisotropy also playing a significant role. These findings show that the development of a $^{13}$CF$_3$ TROSY experiment is a worthwhile pursuit, which could extend the scope of NMR.\\
Validation of this theory was achieved via studies on a small $^{13}$CH$_3$-labelled molecule, present in a highly viscous medium. Although our theory described its relaxation well, the molecule was not effective at mimicking proteins.\\
\end{summarytext}


\AtNextBibliography{\footnotesize}
%\begin{singlespacing}
%\printsummarybibliography
%\end{singlespacing}
